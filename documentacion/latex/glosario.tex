\newacronym{aws}{AWS}{Amazon Web Services}
\newacronym{s3}{S3}{Simple Storage Solution}
\newacronym{api}{API}{Application Programming Interface}
\newacronym{url}{URL}{Uniform Resource Locator}
\newacronym{npm}{NPM}{Node Package Manager}
\newacronym{rest}{REST}{REpresentational State Transfer}
\newacronym{saas}{SaaS}{Software as a Service}
\newacronym{bbdd}{BBDD}{Bases de Datos}
\newacronym{sspl}{SSPL}{Server-Side Public Licence}
\newacronym{mpl}{MPL}{Mozilla Public Licence}
\newacronym{gpl}{GPL}{General Public Licence}
\newacronym{agpl}{AGPL}{Affero General Public Licence}
\newacronym{aks}{AKS}{Azure Kubernetes Service}
\newacronym{aci}{ACI}{Azure Container Instance}
\newacronym{aca}{ACA}{Azure Container Apps}
\newacronym{fqdn}{FQDN}{Fully Qualified Domain Name}
\newacronym{cors}{CORS}{Cross-Origin Resource Sharing}
\newacronym{cli}{CLI}{Command Line Interface}
\newacronym{tui}{TUI}{Terminal User Interface}
\newacronym{iac}{IaC}{Infrastructure as Code}
\newacronym{crud}{CRUD}{Create Read Update Destroy}
\newacronym{tls}{TLS}{Transport Layer Security}
\newacronym{json}{JSON}{JavaScript Object Notation}
\newacronym{hcl}{HCL}{Hashicorp Configuration Language}
\newacronym{ip}{IP}{Internet Protocol}
\newacronym{cicd}{CI/CD}{Continuous Integration and Continuous Deployment}
\newacronym{yaml}{YAML}{YAML Ain't Markup Language}
\newacronym{pr}{PR}{Pull Request}
\newacronym{ssh}{SSH}{Secure SHell}
\newacronym{arm}{ARM}{Azure Resource Manager}
\newacronym{cdk}{CDK}{Cloud Development Kit}
\newacronym{https}{HTTPS}{HyperText Transfer Protocol Secure}
\newacronym{acr}{ACR}{Azure Container Registry}

\newglossaryentry{azure}{
    name={Azure},
    description={La plataforma cloud de Microsoft}
}
\newglossaryentry{frontend}{
    name={Frontend},
    description={Servidor de archivos con los que un navegador puede renderizar una página web}
}
\newglossaryentry{cliente}{
    name={Cliente},
    description={Proceso que se conecta a un servidor, en el caso de una página web, el código de la página que corre en navegador del usuario}
}
\newglossaryentry{middleware}{
    name={Middleware},
    description={Servidor entre frontend y bases de datos}
}
\newglossaryentry{backend}{
    name={Backend},
    description={Servidor final, con la lógica de negocio. En esta practica, intercambiable con middleware}
}
\newglossaryentry{bucket}{
    name={Bucket},
    description={Unidad de almacenamiento en AWS}
}
\newglossaryentry{blob}{
    name={Blob},
    description={Unidad de almacenamiento en los contenedores de almacenamiento de Azure}
}
\newglossaryentry{url_prefirmada}{
    name={URL pre-firmada},
    description={URL con credenciales de uso limitado incluidos, para distribución}
}
\newglossaryentry{http_get}{
    name={HTTP GET},
    description={Método de HTTP en el que se pide un recurso}
}
\newglossaryentry{http_put}{
    name={HTTP PUT},
    description={Método de HTTP en el que se pide crear un recurso. Es idempotente}
}
\newglossaryentry{http_post}{
    name={HTTP POST},
    description={Método de HTTP en el que se pide subir datos. No es idempotente}
}
\newglossaryentry{token}{
    name={Token},
    description={También conocido como identificador de sesión, es una cadena de caracteres con la que seguir a un usuario, y darle acceso sin que tenga que enviar su contraseña repetidas veces}
}
\newglossaryentry{traefik}{
    name={Traefik},
    description={Proxy inverso de última generación, descubre servicios por su cuenta}
}
\newglossaryentry{reverse_proxy}{
    name={Reverse-Proxy},
    description={Proxy dentro del firewall, en particular usado por recursos dentro de la organización para acceder a otros también dentro. En mi caso, el valor es presentar una única dirección, saltándome el problema de CORS}
}
\newglossaryentry{wildcard}{
    name={Wildcard},
    description={Valor que representa una equivalencia a cualquier valor}
}
\newglossaryentry{infra}{
    name={Infraestructura},
    description={Conjunto de componentes hardware y software sobre los que se apoya una aplicación}
}
\newglossaryentry{mongodb}{
    name={MongoDB},
    description={Base de datos no-SQL documental}
}
\newglossaryentry{redis}{
    name={Redis},
    description={Base de datos llave-valor solo guardad en memoria. Rápida pero no persistente, para tokens}
}
\newglossaryentry{fullstack}{
    name={Fullstack},
    description={Conjunto de todos los sistemas, frontend, backend y bases de datos}
}
\newglossaryentry{docker}{
    name={Docker},
    description={Software de virtualización por contenedores}
}
\newglossaryentry{graphql}{
    name={GraphQL},
    description={Graph Query Language, alternativa a REST, propuesta por Facebook}
}
\newglossaryentry{onprem}{
    name={On-Prem},
    description={``On-Premises'', Infraestructura en las premisas de la empresa, que no es contratada a terceros}
}
\newglossaryentry{wrapper}{
    name={Wrapper},
    description={Capa de abstracción sobre una herramienta, con la traducción literal ``envoltorio''}
}
\newglossaryentry{cloud}{
    name={Cloud},
    description={Un servicio o plataforma contratado como servicio, el ordenador de un tercero}
}
\newglossaryentry{pipeline}{
    name={Pipeline},
    description={Conjunto de procesos automatizados, normalmente en un repositorio o entorno cloud, que se aplican después de ser disparados por eventos como un commit a dicho repo}
}
\newglossaryentry{overhead}{
    name={Overhead},
    description={Coste añadido incurrido por procesos de gestión o supervisión}
}
\newglossaryentry{localhost}{
    name={Localhost},
    description={El nombre de host estándar para la mísma máquina}
}
